\documentclass[UTF8]{article}
\usepackage{xeCJK,ctex,enumitem,graphicx,amsmath,amssymb,tabularx,supertabular,multirow,makecell,lscape}
\usepackage[squaren]{SIunits}
\usepackage{boxedminipage}
\usepackage[a5paper,left=18mm,right=18mm,top=15mm,bottom=15mm]{geometry}
\usepackage{pifont}
\usepackage[perpage]{footmisc}
\usepackage{fancyhdr}
\pagestyle{fancy}
\fancyhf{}
\cfoot{\thepage}
\title{对冲基金的费用计算}
\author{吟啸且徐行(雨云丘山)}
\date{}
\begin{document}
\maketitle
\section{费用项目介绍}
对冲基金的费用,有时候用“2 and 20”来表示。其中,“2”表示管理费为2\%,“20”表示业绩费为20\%。\\
\indent 基金中基金会额外收取“1 and 10”的费用。
\subsection{管理费}
管理费根据基金规模每年按照固定比例征收,与投资结果无关。可以基于期初或期末。
\subsection{业绩费}
业绩费的计算相对复杂。业绩费基于增额部分收费。它可以基于扣除管理费之后的资产(题目中出现net of management fees),也可以在扣除管理费之前(题目出现management fee and incentive fee are calculated independently)。\\
\indent \textbf{门槛收益率(hurdle rate)}表示拿到激励费的最低回报率,通常基于无风险利率。可以是\textbf{硬门槛},即按照超过门槛的部分收取业绩费;也可以是\textbf{软门槛},基于整个收益收费。\\
\indent 例如,某基金期初管理资金100万元,期末涨到了110万元,门槛收益率为5\%,业绩费为20\%,盈利率\(=\dfrac{110-100}{100}\times 100\%=10\%>5\%\),超过了门槛回报率,因此可以得到业绩费。按照硬门槛,业绩费为\((110-100\times 1.05)\times 0.2=1\)万元;按照软门槛,业绩费为\(110-100)\times 0.2=2\)万元。\\
\indent \textbf{高水位线(high water rate)}表示基金的历史最高水平。只有超过高水位线,才能以超出部分为基准计算业绩费。高水位线的目的是避免对重复的业绩二次收费。
\subsection{费用谈判}
有限合伙人和一般合伙人谈判确定基金费用。一般情况下,投资期限越长,费率越低。
\section{题目考法}
例题来自于CFA\(^{\textregistered}\)官方教材与练习平台。\\
\indent \textbf{【例题1】}The following information is available about a hedge fund:\\
\begin{center}
\begin{tabular}{|m{5cm}|m{4cm}|} \hline
Initial fund assets & \$100 million \\ \hline
Fund assets at the end of the period (before fees)& \$110 million \\ \hline
Management fee based on assets under management& 2\% \\ \hline
Incentive fee based on the return& 20\% \\ \hline
Soft hurdle rate& 8\% \\ \hline
\end{tabular}
\end{center}
\indent No deposits to the fund or withdrawals from the fund occurred during the year. Management fees are calculated using end-of-period valuation. Management fees and incentive fees are calculated independently. The net-of-fees return of the investor is closest to: \\
\indent A.7.8\%.\\
\indent B.7.4\%.\\
\indent C.5.8\%.\\
\indent 【解答】首先判断有没有业绩费,初始管理资产为100万美元,期末毛收入为110万美元,超过了hurdle rate(门槛收益率),因此是有业绩费的。\\
\indent 题目中说“Management fees and incentive fees are calculated independently”,因此业绩费的计准是扣除管理费之前的盈利。又因为本题为软门槛,因此业绩费基于全部盈利进行计算。\\
\indent 综上,投资者净收益为\(\text{总收益}-\text{管理费}-\text{业绩费}=110-110\times0.02-(110-100)\times0.2=5.8\),收益率为\(\dfrac{5.8}{100}=5.8\%\)。本题选C。\\
\indent \textbf{【例题2】}An investor allocates \$10 million at the beginning of the year to a hedge fund charging a management fee of 2\% and an incentive fee of 20\% with a 6\% (hard) hurdle rate. At year-end the value of the investment is \$11.8 million. The incentive fee is calculated net of the management fee and the management fee is based on the year-end value. The net-of-fees return the investor earned is closest to: \\
\indent A.13.71\%. \\
\indent B.13.24\%. \\
\indent C.13.93\%. \\
\indent 【解答】管理费基于年末价值计算,其值为\(11.8\times0.02=0.236\)百万美元。\\
\indent 业绩费的部分,首先判断有没有业绩费;业绩费基于扣除管理费的净值,扣除管理费后的收益为\(11.8\times(1-0.02)=15.64>10.6\),超过了硬门槛,因此有业绩费。基于高出硬门槛的部分计算业绩费得\([11.8\times(1-0.02)-10.6]\times0.2=0.1936\)。因此,投资者净收益为\(\dfrac{11.8-10-0.236-0.1936}{10}=13.70\%\),正确答案为A。
\section{总结}
计算对冲基金费用一般可以按照以下几个步骤进行:
\begin{enumerate}[fullwidth,itemindent=3em]
\item 计算管理费。
\item 判断有无业绩费。
\item 识别业绩费计算方式并计算业绩费。
\item 注意题目问的是费用还是净收益。
\item 如果题目中涉及基金中基金,很可能会有额外的收费。
\end{enumerate}
\end{document}